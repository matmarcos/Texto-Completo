\documentclass[12pt,a4paper]{amsart}
%\usepackage[brazil]{babel}
%\usepackage[portuguese]{babel}
\usepackage[T1]{fontenc}
\usepackage{indentfirst}
\usepackage{graphicx}
\usepackage[hmargin={2cm},vmargin={2cm}]{geometry}
\usepackage{graphicx,color}

 
\usepackage{ifthen}                            % Caso tenha problemas ao usar UTF-8, experimente \usepackage{ucs}
\usepackage{calc}


\usepackage{amssymb,amsmath}   % simbolos matem�ticos providos pela AMS
\usepackage{amsthm}						 % estilo dos teoremas
%\usepackage[dvipdfm]{graphicx} % para inclus�o de figuras (png, jpg, gif, bmp)
\usepackage{fancyhdr}					 % personaliza cabe�alho e rodap�
\usepackage{color}             % para letras e caixas coloridas
%\usepackage{makeidx}           % �ndice remissivo
\usepackage{a4wide}            % correta formata��o da p�gina em A4
\usepackage{setspace}          % para a dist�ncia entre linhas
\usepackage[T1]{fontenc}
%\usepackage{egothic}
%\usepackage{yfonts}
\usepackage[all]{xy}
\usepackage{stackrel}	
\usepackage{amsmath}
\usepackage{geometry}
\usepackage{mathabx}

\parskip 5pt

\newcommand{\C}{\ensuremath{\mathbb{C}}}
\newcommand{\dem}{{\bf Demonstra��o: }}
\newcommand{\R}{\ensuremath{\mathbb{R}}}
\newcommand{\Q}{\ensuremath{\mathbb{Q}}}
\newcommand{\F}{\ensuremath{\mathbb{F}}}
\newcommand{\I}{\ensuremath{\mathbb{I}}}
\newcommand{\Z}{\ensuremath{\mathbb{Z}}}
\newcommand{\N}{\ensuremath{\mathbb{N}}}
\newcommand{\Rm}{\ensuremath{\R^m}}
\newcommand{\Rn}{\ensuremath{\R^n}}
\newcommand{\RN}{\ensuremath{\R^N}}
\newcommand{\oo}{\ensuremath{\infty}}
\newcommand{\omg}{\ensuremath{\Omega}}
\newcommand{\bola}{\circ}
\newcommand{\vazio}{\emptyset}
\newcommand{\x}{\mathcal{X}}
\newcommand{\cqd}{\rule{2mm}{2mm}}
\newcommand{\prova}{\noindent \textbf{Proof: }}

\renewcommand{\S}{\ensuremath{{\mathcal{S}}}}
\newcommand{\E}{\ensuremath{{\mathcal{E}}}}
\newcommand{\D}{\ensuremath{{\mathcal{D}}}}
\newcommand{\Coo}{\ensuremath{{C}^\oo}}
\newcommand{\Ck}[1][1]{\ensuremath{{C}^{#1}}}
\newcommand{\lloc}[1][1]{\ensuremath{L^#1_{\mbox{\emph{\scriptsize{loc}}}}}}
\renewcommand{\l}[1][1]{\ensuremath{L^{#1}}}

\newcommand{\norma}[2][ ]{\ensuremath{|| #2 ||_{#1}}}
\newcommand{\seta}{\longrightarrow}
\newcommand{\leva}{\longmapsto}
\newcommand{\eps}{\ensuremath{\varepsilon}}
\DeclareMathOperator{\sen}{sen}         
\DeclareMathOperator{\senh}{senh} 
\DeclareMathOperator{\mdc}{mdc} 
\DeclareMathOperator{\mmc}{mmc} 
\DeclareMathOperator{\supp}{supp} 



\renewcommand{\d}[1]{\ensuremath{{\rm d} #1 }}
\newcommand{\intg}[3][ ]{\ensuremath{ \int_{#1} #2 \, \d{#3}}}
\newcommand{\intx}[2][ ]{\intg[#1]{#2}{x}}
\newcommand{\intxi}[2][ ]{\intg[#1]{#2}{\xi}}
\newcommand{\tf}[1]{\ensuremath{\widehat{#1}}}
%\newcommand{\txt}[1]{\mbox{ #1 }}
\newcommand{\til}[1]{\widetilde{#1}}

\newcommand{\der}[1]{{\sf Der}(#1)}
\newcommand{\gl}[1]{{\mathfrak g\mathfrak l}(#1)}
\newcommand{\ann}[2]{{\sf Ann}_{#1}(#2)}

\newtheorem{teo}{Theorem}[section]
\newtheorem{cor}[teo]{Corollary}
\newtheorem{lem}[teo]{Lemma}
\newtheorem{obs}[teo]{Observation}

\newtheorem{af}[teo]{Afirma��o}
\newtheorem{prop}[teo]{Proposition}

\theoremstyle{definition}
\newtheorem{df}[teo]{Definition}
\newtheorem{ex}[teo]{Example}

\newcommand{\ad}[2]{{\sf ad}^{#1}_{#2}}
\newcommand{\ctwo}[2]{{\sf C}^2(#1,#2)}
\newcommand{\ztwo}[2]{{\sf Z}^2(#1,#2)}
\newcommand{\btwo}[2]{{\sf B}^2(#1,#2)}
\newcommand{\htwo}[2]{{\sf H}^2(#1,#2)}
\newcommand{\zone}[2]{{\sf Z}^1(#1,#2)}
\renewcommand{\hom}[2]{{\sf Hom}(#1,#2)}
\newcommand{\comp}[2]{{\sf Comp}(#1,#2)}
\newcommand{\sdsum}{\oright}
\newcommand{\ennd}[1]{{\sf End}(#1)}

\makeindex

\begin{document}

\title[Non-singular derivations in prime characteristic]
      {Derivations of Lie algebra extensions and non-singular
derivations of Lie algebras in prime characteristic}

\author{Marcos Lima Goulart}
\address{UFOP}
\author{Csaba Schneider}
\address{Departamento de Matem\'atica\\
Instituto de Ci\^encias Exatas\\
Universidade Federal de Minas Gerais\\
Av.\ Ant\^onio Carlos 6627\\
Belo Horizonte, MG, Brazil\\
csaba@mat.ufmg.br\\
www.mat.ufmg.br/$\sim$csaba}

\maketitle

\section{Introduction}

Jacobson's famous theorem states that a finite-dimensional Lie algebra
over a field of characteristic zero that admits a non-singular derivation must
be nilpotent. It is well-known that this theorem is not valid when 
the characteristic is non-zero. Non-nilpotent and solvable examples
were constructed by Shalev~\cite{} and Mattarei~\cite{}, whereas
the simple Lie algebras with non-singular derivations were classified 
by Benkart and her collaborators in~\cite{}. 

Despite the existing examples, little is known about non-nilpotent Lie algebras
with non-singular derivations. In these notes we explore the structure of
solvable, non-nilpotent Lie algebras with non-singular derivations. 
In order to study derivations of solvable Lie algebras, we develop a theory
of the derivations of Lie algebra extensions. We adopt the concept of a compatible pair 
of automorphisms introduced in~\cite{} for derivations of Lie algebras. 


\section{Lie algebra extensions}
The symbol `$\oplus$' will be used to denote the direct
sum of algebras,
while  the
direct sum of vector spaces will be denoted by `$\dotplus$'.


An {\em extension} of a Lie algebra  $K$ by a Lie algebra
$I$  is an exact sequence 
\begin{equation}\label{ext}
0 \to I \stackrel{ i}{\to} L \stackrel{ s}{\to} K \to 0
\end{equation}
of Lie algebras.
The Lie algebra $L$ in the middle of the exact sequence contains an
ideal
$\ker s=\mbox{Im}\,i\cong I$ such that $L/I\cong K$. We will write
informally that
`$L$ is an extension of $K$ by $I$'. The extension~\eqref{ext}
{\em splits}
if $L$ has a subalgebra $S$ such that $L=S \dotplus\ker s$. 
The extension~\eqref{ext} is {\em trivial} if there exists an ideal $S$ of $L$
such that $L=S\oplus\ker s$. The extension~\eqref{ext} is central if 
$\ker s$ lies in the center $Z(L)$ of $L$.

Suppose that $K$ and $I$ are Lie algebras and $\psi:K\rightarrow\der I$
is a given Lie algebra homomorphism.
Then we say that  $K$ {\em acts} on $I$ or that $I$ is a {\em $K$-module}.
In this case, the image $\psi(k)(a)$ of $a\in I$ under
$k\in K$ will be written
simply as $[k,a]$.
If $I$ is an ideal of a Lie algebra $K$, then $K$ acts on $I$. If 
$k\in K$, then the image of $k$ under this action will be denoted by $\ad Ik$ or
simply by $\ad {}k$ when the domain of the representation is clear from 
the context. Thus, for $a\in I$ and for
$k\in K$, $\ad Ik(a)=\ad{}k(a)=[k,a]$.  The homomorphism
$K\rightarrow \der I$ that takes $k\mapsto\ad Ik$, 
will be denoted by $\ad I{}$. 

\begin{ex} Let $L$ be a Lie algebra with an abelian ideal $I$ and
  set $K=L/I$.
  Define the Lie algebra representation $\ad{I}{}:K \to \der I$
  by $\ad{I}{x+I}(a)=[x,a]$ for
  all $x\in L$ and $a \in I$. This is well defined, since
  $I$ is abelian. Then $I$ is a $K$-module. In this case,
  we say that the action is {\em induced by the
    adjoint representation}.
\end{ex}



Let $K$ be a Lie algebra over a field $\F$
and let $I$ be a vector space over $\F$.
Denote by $\ctwo KI$ the vector space of alternating bilinear maps
$\vartheta: K \times K \to I$.
If $I$ is a $K$-module and
$\vartheta \in \ctwo KI$ has the property that
$$
\vartheta(x,[y,z])+\vartheta(y,[z,x])+\vartheta(z,[x,y])=
[x,\vartheta(y,z)]+[y,\vartheta(z,x)]+[z,\vartheta(x,y)],
$$
for all $ x,\ y,\ z \in K$, then $\vartheta$ is said to be a
{\em cocycle} and the vector space of coclycles is denoted by $\ztwo KI$.
Let $T:K \to I$ be a linear transformation and define, 
$\vartheta_T:K\times K\rightarrow I$ by
$$
\vartheta_T(k,h)=T([h,k])+[k,T(h)]-[h,T(k)]\quad
\mbox{for all}\quad k,\ h \in K.
$$
Then $\vartheta_T\in\ztwo KI$ and such
a cocycle $\vartheta_T$ is said to be
a {\em coboundary}.
The set of coboundaries is denoted by $\btwo KI$. 
The set $\btwo KI$ is a subspace of
$\ztwo KI$, and we set $\htwo KI=\ztwo KI/\btwo KI$ to be
the quotient space. 
The first cohomology group of $K$ and $I$ is defined as
$$\zone KI=\{\nu \in \hom KI \mid \nu([k,h])=[h,\nu(k)]-[k,\nu(h)]
\mbox{ for all } k,\ h \in K\}.
$$


The next result, whose proofs can be found, for instance, 
in \cite[Section~4.2]{Knapp}, links Lie algebra extensions to cohomology.
Let $K$ be a Lie algebra and let $I$ be a $K$-module.
  Let $\vartheta \in \ztwo KI$ and
  define the Lie algebra $K_{\vartheta}=K \dotplus I$ with the product
  \begin{equation}
 [x+a,y+b]=[x,y]+\vartheta(x,y)+[a,y]-[b,x] \mbox{ for all } x,\ y \in K \mbox{ and } a,\ b \in I.
 \end{equation}
 
\begin{prop} \label{prop:cocic}
 The following hold for the Lie algebra $K_\vartheta$:
 \begin{enumerate}
 \item $K_\vartheta$ is a Lie algebra extension of $K$ by $I$;
 \item if $\nu\in\btwo KI$, then
   $K_\vartheta$ is isomorphic to $K_{\vartheta+\nu}$;
 \item  if $\vartheta\in\btwo KI$, then
   $K_\vartheta$ is a split extension of $K$ by $I$.
 \end{enumerate}
 Conversely, let  $L$ be a Lie algebra and $J$ be an abelian ideal of
 $L$. Then there exists $\vartheta \in \ztwo{L/J}J$ such that
 $L \cong (L/J)_\vartheta$. 
\end{prop}

The cocycle $\vartheta$ in last statement   of
Proposition~\ref{prop:cocic} can
be constructed as follows. Let $\pi:L\rightarrow L/I$ denote the
natural projection, and let $\sigma: L/I\rightarrow L$ be
a right inverse of $\pi$; that is,
$\pi\sigma=\mbox{id}_{L/I}$. Then, for $k+I,\ h+I\in L/I$,
set
$$
\vartheta(k+I,h+I)=\sigma([k+I,h+I])-[\sigma(k+I),\sigma(h+I)].
$$
Routine calculation shows that $\vartheta\in\ztwo {L/I}I$ and
that $L\cong L_\vartheta$.
    

 
\section{Compatible pairs and derivations of subdirect sums}

Compatible pairs were introduced in~\cite{} to compute automorphisms
of solvable groups and solvable Lie algebras. We adopt the
concept for derivations of Lie algebras.  
Let $K$ and $I$ be Lie algebras such that $K$ acts on $I$
via the homomorphism $\psi:K\rightarrow\der I$.
We  define the {\em semidirect sum} $K\sdsum_\psi I$
as the vector
space $K\dotplus I$ with the product operation given as
$$
[(k_1,a_1),(k_2,a_2)]=([k_1,k_2],[k_1,a_2]-[k_2,a_1]+[a_1,a_2]).
$$
When the $K$-action on $I$ is clear from the context, then we
usually suppress the homomorphism `$\psi$' from the notation and write
simply $I\sdsum K$. If $L$ is a Lie algebra such that $L$ has
an ideal $I$ and a subalgebra $K$ in such a way that
$L=K\dotplus I$, then $L\cong K\sdsum_\psi I$ where $\psi$
is the restriction of $\ad {}I$ to $K$.
In a semidirect sum $K\sdsum I$, an element $(k,a)\in K\dotplus I$
will usually be written as $k+a$. 


Suppose that $K$ and $I$ are as in the previous paragraph.
The direct sum $\der K\oplus\der I$ of
the derivation Lie algebras is a Lie algebra.
An element $(\alpha,\beta)\in \der K\oplus\der I$ is said to be a
{\em compatible pair} if
\begin{equation}{\label{compeq}}
  \beta ([k,a])=[\alpha(k),a]+[k,\beta(a)] \quad\mbox{for all}\quad
  k \in K,\ a \in I.
\end{equation}
We let $\comp KI$ denote the set of compatible pairs in
$\der K\oplus\der I$.
Using the homomorphism $\psi:K\rightarrow\der I$
associated to the $K$-action on $I$, we can write equation (\ref{compeq}) in another form
as follows. 
Writing $[k,a]$ as $\psi(k)(a)$, we have that
$(\alpha,\beta) \in \comp KI$ if and only if the equation
\begin{equation*}
  \beta \psi(k)= \psi (\alpha(k))+\psi(k)\beta.
\end{equation*}
holds in $\der I$ for all  $k \in K$.
Using commutator, this is equivalent to  
\begin{equation}{\label{compcomu}}
  [\beta,\psi(k)]=\psi (\alpha(k)) \mbox\quad \mbox{for all} \quad k \in K.
\end{equation}
Letting $\ad{}{}:\der I\to \der I$ denote the 
adjoint representation, equation~\eqref{compcomu} can be rewritten
as
$$
\ad{}{\beta}\psi(k)=\psi (\alpha(k))\quad \mbox{for all}\quad k \in K.
$$
Therefore, $(\alpha,\beta) \in\comp KI$
if and only if the following diagram commutes:
$$ \xymatrix{K \ar[d]^{ \alpha} \ar[r]^{\psi} \ar@{}[dr]|{\circlearrowright} &
  {\der I} \ar[d]^{\ad{}{\beta}} \\ K \ar[r]^{\psi} & {\der I}.} $$

A compatible pair $(\alpha,\beta)\in
\der K\oplus\der I$ will usually be written as $\alpha+\beta$.
If $\alpha+\beta\in\der K\oplus\der I$ as above,
then $\alpha+\beta$ can be considered a element of
$\gl{I\oplus K}$ by letting $(\alpha+\beta)(a+k)=
\alpha(a)+\beta(k)$ for all $a\in I$ and $k\in K$.

\begin{prop}
  Using the notation above,
  we have that
  $$
  \comp KI=\{\alpha+\beta\in\gl K\oplus\gl I\mid
  \alpha+\beta\in\der{K\sdsum I}\}.
  $$
  In particular $\comp KI$ is a Lie subalgebra of $\der{K\sdsum I}$.
\end{prop}
\begin{proof}
%%   First we show that $\comp IK$ is 
%%   a Lie subalgebra of  $\der I\oplus
%%   \der K$.
%%   Suppose that $(\alpha,\beta),\ (\alpha',\beta') \in \comp KI$ and
%% assume that the $K$-action on $I$ is given by the representation
%% $\psi:K \to \der I$ such that $\psi(k)(a)=[a,k]$ for all
%% $k \in K$ and $a \in I$. First we check that
%% $\comp KI$ is a vector subspace using equation (\ref{compcomu}).
%% If $\lambda \in \F$ and $k \in K$ then
%% $$
%% \begin{array}{rcl}
%% [\psi(k),\beta+\lambda \beta'] & = & [\psi(k),\beta]+ \lambda[\psi(k), \beta'] \\
%%  & = & -\psi(k)(\alpha)-\lambda\psi(k)(\alpha') \\
%%   & = & \psi(k)(\alpha+\lambda\alpha').
%% \end{array}
%% $$
%% Hence $(\alpha,\beta)+\lambda(\alpha',\beta') \in \comp KI$.

%% Using the definition of compatible pairs, 
%% $$
%% \beta' \psi(k)=\psi(k)\beta'+\psi(\alpha'(k)).
%% $$ 
%% Then
%% $$
%% \begin{array}{rcl}
%% \beta \beta' \psi(k) & = & \beta\psi(k)\beta'+\beta \psi (\alpha'(k)) \\
%% &=& \psi(k)\beta\beta'+\psi (\alpha(k))\beta'+\psi(\alpha'(k)) \beta+ \psi(\alpha'\alpha(k)).
%% \end{array}
%% $$
%% Similarly,
%% $$
%% \beta'\beta \psi(k)=\psi(k)\beta'\beta+\psi (\alpha'(k))\beta+\psi(\alpha(k)) \beta'+ \psi(\alpha\alpha'(k)).
%% $$
%% Therefore
%% $$
%% [\beta,\beta']\psi(k)=\psi(k)(\beta\beta'-\beta'\beta)+\psi((\alpha \alpha' - \alpha' \alpha)(k))=\psi(k)[\beta,\beta']+\psi([\alpha, \alpha'](k)).
%% $$
%% Hence $[(\alpha,\beta),(\alpha',\beta')] \in \comp KI$.


  Suppose that $\alpha+\beta\in\comp KI$ is a compitabile pair
  and let $k+a,\ k'+a'\in K\sdsum I$.
  Then 
\begin{multline*}
  (\alpha+\beta)[k+a,k'+a'] =  (\alpha+\beta)
  ([k,k']+([k,a']-[k',a]+[a,a']))\\
  = \alpha([k,k'])+\beta([k,a']-[k',a]+[a,a'])\\
 =[\alpha(k),k']+[k,\alpha(k')]+[\alpha(k),a']-[\alpha(k'),a]+
  [\beta(a),a']+[k,\beta(a')]-[k',\beta(a)]+[a,\beta(a')].
  \end{multline*}
On the other hand
\begin{multline*}
[(\alpha+\beta)(k+a),k'+a']+[k+a,(\alpha+\beta)(k'+a')]=\\
[\alpha(k),k']+[\alpha(k),a']+[\beta(a),k']+[\beta(a),a']+
[k,\alpha(k')]+[k,\beta(a')]+[a,\alpha(k')]+[a,\beta(a')].
\end{multline*}
Thus $\alpha+\beta\in\der{K\sdsum I}$.

Conversely, let
$\alpha+\beta\in \gl{K}\oplus \gl I$ such that
$\alpha+\beta$ is a derivation of $K\sdsum I$. Then
$(\alpha+\beta)|_K=\alpha$ and $(\alpha+\beta)|_I=\beta$, and so
$\alpha\in\der K$ and $\beta\in\der I$. Further, if $k\in K$ and
$a\in I$, then $[k,a]\in I$, and so
$$
\beta([k,a])=(\alpha+\beta)[k,a]=[(\alpha+\beta)(k),a]+
     [k,(\alpha+\beta)(a)]=[\alpha(k),a]+[k,\beta(a)].
     $$
     Thus $\alpha+\beta\in\comp KI$, as required.

     The fact that $\comp KI$ is a Lie subalgebra of $\der{K\sdsum I}$
     follows from the fact that
     $\comp KI$ is the intersection of two Lie algebras; namely,
     $\comp KI=(\gl K\oplus\gl I)\cap\der{K\sdsum I}$.
\end{proof}




Let $K$ and $I$ be vector spaces. Consider the Lie algebra
$\gl K\oplus\gl I$ and define an action of  $\gl K\oplus\gl I$
on the vector space $\hom K{\gl I}$ as follows.
Let $\ad{}{}$ denote the adjoint representation
of $\gl I$. Thus, for $\beta,\ \beta'\in\gl I$ and $\ad{}\beta(\beta')=
[\beta,\beta']$.
For $\alpha+\beta \in \gl K
\oplus \gl I$ and for $T \in \hom K{\gl I}$, set
 \begin{equation}{\label{acaocomp}}
 (\alpha+\beta)\cdot T=\ad{}{\beta}T-T\alpha.
 \end{equation}
 Let us show that this in fact defines a Lie algebra
 action.
 First notice that
 $(\alpha,\beta) \cdot T$ is a linear map because
 is linear combination of composition and sums of linear maps.
 Let us check that it preserves Lie brackets.
 Let $\alpha+\beta,\ \alpha'+\beta'
 \in \gl K \oplus \gl I$
 and $k \in K$. By definition
 $$
 (\alpha',\beta')\cdot T=\ad{}{\beta'}T-T\alpha'.
 $$
 So
 $$(\alpha,\beta)\cdot((\alpha',\beta')\cdot T)=\ad{}{\beta}\ad{}{\beta'}T-\ad{}{\beta'}T\alpha-\ad{}{\beta}T \alpha'+T\alpha'\alpha.$$
 In the same way, 
$$(\alpha',\beta')\cdot((\alpha,\beta)\cdot T)=\ad{}{\beta'}\ad{}{\beta}T-\ad{}{\beta}T\alpha'-\ad{}{\beta'}T \alpha+T\alpha\alpha'.$$
Hence, $$\begin{array}{rcl}
(\alpha,\beta)\cdot((\alpha',\beta')\cdot T)-(\alpha',\beta')\cdot((\alpha,\beta)\cdot T) & = &\ad{}{\beta}\ad{}{\beta'}T-\ad{}{\beta'}\ad{}{\beta}T+T\alpha\alpha'-T\alpha'\alpha \\ 
&=& [\ad{}{\beta},\ad{}{\beta'}]T+T[\alpha,\alpha'].
\end{array}$$
Therefore, $$[(\alpha,\beta),(\alpha',\beta')]\cdot T= ([\alpha,\alpha'],[\beta,\beta'])\cdot T.$$

Now, if $K$ is a Lie algebra and $I$ is a $K$-module, then
there is a corresponding homomorphism
$\psi\in\hom K{\der I}$. Now suppose that $\alpha+\beta\in \gl K\oplus
\gl I$ such that $\alpha+\beta\in\der K\oplus\der I$. 
Then, for $k \in K$, we have $\ad{}{\beta}T(k)+T\alpha(k)$ is a derivation of $I$ since $\ad{}{\beta}T(k),\ T\alpha(k) \in\der I$.

If $X$ is a subalgebra of $\der K\oplus\der I$, then 
the annihilator $\ann K\psi$ of $\psi$ in $X$ is defined as
$$
\ann X\psi=\{(\alpha,\beta)\in X\mid
(\alpha,\beta)\cdot \psi=0\}.
$$
Computing the annihilator of $\psi$ in $\der K\oplus \der I$
explicitly, we obtain
\begin{multline*}
\ann{\der K\oplus \der I}\psi  =  \{(\alpha,\beta) \in \der K\oplus \der I \mid (\alpha,\beta)\cdot \psi=0 \} \\
= \{(\alpha,\beta) \in \der K\oplus \der I \mid  \ad{}{\beta}\psi-\psi\alpha=0 \} 
= \comp KI.
\end{multline*}
The last equality follows from~\eqref{compcomu}.
Hence we have  proved the following  proposition.

\begin{prop} Let $K$ and $I$ be Lie algebras such that $I$
  is also a $K$-module via 
  the representation
  $\psi\in\hom K{\der I}$. Then
  $\comp KI=\ann{\der K\oplus\der I} \psi$, where the
  action of $\der K\oplus \der I$ on  $\hom K{\der I}$
  is given by~\eqref{acaocomp}.
\end{prop}


 
 
 \section{Nilpotent Subalgebras of Compatible Pairs}

 \begin{prop}  Let $V$ be a finite dimensional vector space over a algebraically closed field $\F$ and $D \subseteq \mathfrak{gl}(V)$ a nilpotent linear algebra. Then $V$ has a unique decomposition $V=V_{\lambda_1} \oplus \cdots \oplus V_{\lambda_n}$ into $D$-modules such that $$V_{\lambda_i}=\{v \in V \mid  \mbox{ for all } d \in D \mbox{ there is an } m>0  \mbox{ such that } (d - \lambda(k))^mv=0\},$$
 where $\lambda_i:D \to \F, 1 \leq i \leq n$. The space $V_{\lambda_i}$ is called a generalized eigenspace of $D$ with eigenvalue $\lambda_i$. 
 \end{prop}

\prova A proof of this fact can be found in chapter 3 of \cite{deGraaf}, for example.

\vspace{0,5cm}

 Let $K$ and $I$ be Lie algebras such that $K$ act on $I$. If $D \subseteq Comp(K,I)$ is nilpotent subalgebra then by \textbf{Proposition \ref{compsubgl}}, $D$ can be seen as subalgebra of derivations of semidirect sum $L=K\oplus I$. The decomposition of $L$ in eigenspaces of $D$ induces decompositions in $K$ and $I$, because as subspaces of $L$ they are invariants by $D$. So each nilpotent subalgebra of $Comp(K,I)$ induces unique decompositions in generalized eigenspaces of in $K$ and $I$.
 
 
 \begin{prop} Let $K$ and $I$ be Lie algebras such that $K$ act on $I$. If $(\alpha, \beta) \in Comp(K,I)$ then \begin{equation}{\label{comp2}}
(\beta-(\lambda+\mu))^n[a,k]=\sum_{i=0}^n\binom{n}{i}[(\beta-\lambda)^{n-i}(a),(\alpha-\mu)^i(k)] \mbox{ for all } a \in I,k \in K \mbox{ and } \lambda,\mu \in \F
\end{equation}


 \end{prop}{\label{3.1}}
\prova Suppose that $\F$ is the base field. We will proof this result by induction on $n$. If $n=1$ the result follow by compatible pair definition. 
Suppose that the result is valid for $n>0$. then

\begin{multline}
(\beta-(\lambda+\mu))^{n+1}[a,k]  =  (\beta-(\lambda+\mu))\sum_{i=0}^n\binom{n}{i}[(\beta-\lambda)^{n-i}(a),(\alpha-\mu)^i(k)] \\
  = \sum_{i=0}^n\binom{n}{i} \left( [(\beta-\lambda)^{n+1-i}(a),(\alpha-\mu)^i(k)] 
  +[(\beta-\lambda)^{n+1-i}(a),(\alpha-\mu)^{i+1}(k)] \right )  \\
  =\sum_{i=0}^{n+1}\binom{n+1}{i}[(\beta-\lambda)^{n+1-i}(a), \mbox{ for all } a \in I,k \in K \mbox{ and } \lambda,\mu \in \F
\end{multline}

\begin{flushright}
 \cqd
 \end{flushright} 
 
  \begin{prop} Let $K$ and $I$ be Lie algebras over an algebraically closed field such that $K$ act on $I$. Let $D$ be a nilpotent subalgebra of $Comp(K,I)$. If $\lambda,\mu:D \to \F$ are generalized eigenvalues of $D$, respectively, then $[I_{\mu},K_\lambda] \subseteq I_{\lambda+\mu}$ if $\lambda+\mu$ is a generalized eigenvalue of $D$. Otherwise $[I_{\mu}, K_{\lambda}]=0$.
\end{prop}

\prova Let $a \in I_{\mu}$, $k \in K_{\lambda}$ and $d \in D$ then by \textbf{Proposition \ref{3.1}} $$(\beta-(\lambda(d)+\mu(d)) I)^n[a,k]=\sum_{i=0}^n\binom{n}{i}[(\beta-\lambda(d) I)^{n-i}(a),(\alpha-\mu(d) I)^i(k)].$$ And for $n$ big enough this is 0. So, $[I_{\mu_i}, K_{\lambda_j}] \subset I_{\mu_i+\lambda_j}$ if $\mu_i+\lambda_j$ is generalized eigenvalue of $D$, otherwise $[I_{\mu_i}, K_{\lambda_j}]=0$ is nonsingular and it follows that $(\beta-(\lambda+\mu))=0$.

\begin{flushright}
 \cqd
\end{flushright}

\begin{prop}{\label{rowen}} Suppose $R$ is a finite dimensional algebra over a field $\F$. If $S$ is a multiplicatively closed subset each of whose elements is a sum of nilpotent elements then $S$ is nilpotent.
\end{prop}
\prova \cite{Rowen}, \textbf{Proposition 2.6.32} pg 178.

\begin{flushright}
 \cqd

\end{flushright}

\begin{teo} Let $K$ and $I$ be finite dimensional Lie algebras over an algebraically closed field $\F$ such that $K$ act on $I$ by representation $\psi:K \to Der(I)$. Let $D$ a nilpotent subalgebra of $Comp(K,I)$ such that zero is not generalized eigenvalue of $D$ in $K$. So if or $char(\F)=0$ or $char(\F)=p$ and dimension of $I$ is smaller than $p$ then $\psi$ is a nilpotent representation.
\end{teo}
\prova  Let $\lambda_1, \cdots, \lambda_r$ and $\mu_1, \cdots, \mu_s \in \F$ be generalized eigenvalue  of $D$ in $K$ and $I$, respectively. If $a \in I_{\mu_i}$ and $k \in K_{\alpha_j}$ then by \textbf{Proposition \ref{3.1}}, we have $(\psi^n(a) \in I_{\mu_i + n \lambda_j}$, with $\lambda_j\neq 0$, if $\mu_i + n \lambda_j$ is generalized eigenvalue of $D$ in $I$ and $(\ad{}{k})^n=0$ otherwise. If $char(\F)=0$ then $(\ad{}{k})^n=0$ for some $n$ because thet set of eigenvalues of $D$ is finite; if $char(\F)=p$ the set $\{\mu_i+ \lambda_j, \mu_i+2 \lambda_j, \cdots, \mu_i+(p-1)\lambda_j,\mu_i\}$ has $p$ distinct elements and $D$ has at most $p-1$ eigenvalues in $I$ then $\psi^n=0$ for some $1 \leq n \leq p$. In both cases  $\psi$ is nilpotent for all $k \in K_{\lambda_j}$, $1 \leq j \leq r$. Hence every element of $\psi(K):K \to Der(I)$ can be written as sum of nilpotent elements. Therefore, by \textbf{Proposition \ref{rowen}}, $\psi$ is nilpotent.  

   \section{Derivations of $K_\vartheta$}
 
 \begin{df} Let $K$ and $I$ be vector spaces. Given $(\alpha,\beta) \in \mathfrak{gl}(K)\oplus \mathfrak{gl}(I)$, $\vartheta \in C^2(K,I)$ and $h,k \in K$, define the action of $\mathfrak{gl}(K)\oplus \mathfrak{gl}(I)$ on $C^2(K,I)$ by \begin{equation}{\label{1}}
 (\alpha,\beta)\cdot \vartheta(h,k)=\beta(\vartheta(h,k))-\vartheta(\alpha(k),h)-\vartheta(k,\alpha(h)).
 \end{equation}
 \end{df}
 
 
\begin{prop}{\label{inva}} Let $K$ be a Lie algebra and $I$ a $K$-module. Considere the action of \newline $Comp(K,I)$ on $C^2(K,I)$ defined in ({\ref{1}}). Then the vector spaces $Z^2(K,I)$ and $B^2(K,I)$ are invariants by this action.
\end{prop}
\prova Let $k,h,l \in K$, $(\alpha,\beta) \in Comp(K,I)$ and $\vartheta \in Z^2(K,I)$. By definition 
$$\begin{array}{rcl}
(\alpha,\beta)\cdot \vartheta(k,[h,l])& = &\beta(\vartheta(k,[h,l]))-\vartheta(\alpha(k),[h,l])-\vartheta(k,\alpha([h,l]))\\ 
& = & \beta(\vartheta(k,[h,l]))-\vartheta(\alpha(k),[h,l])-\vartheta(k,[\alpha(h),l])-\vartheta(k,[h,\alpha(l)]).
\end{array}$$ If $$X=(\alpha,\beta)\cdot\vartheta(k,[h,l])+(\alpha,\beta) \cdot \vartheta(h,[l,k])+(\alpha,\beta)\cdot\vartheta(l,[k,h]),$$ then
\begin{multline*} X=\beta(\vartheta(k,[h,l]))+\beta(\vartheta(h,[l,k]))+\beta(\vartheta(l,[k,h])) \\
 -\vartheta(\alpha(k),[h,l])-\vartheta(\alpha(h),[l,k])-\vartheta(\alpha(l),[k,h]) \\
-\vartheta(k,[\alpha(h),l])-\vartheta(h,[\alpha(l),k])-\vartheta(l,[\alpha(k),h]) \\
-\vartheta(k,[h,\alpha(l)])-\vartheta(h,[l,\alpha(k)])-\vartheta(l,[k,\alpha(h)])
\end{multline*}

Using coclyce definition 
\begin{multline*}
X=\beta([\vartheta(k,h),l])+\beta([\vartheta(h,l),k])+\beta([\vartheta(l,k),h])\\
-[\vartheta(\alpha(k),h),l]-[\vartheta(\alpha(h),l),k]-[\vartheta(\alpha(l),k),h]\\
-[\vartheta(k,\alpha(h)),l]-[\vartheta(h,\alpha(l)),k]-[\vartheta(l,\alpha(k)),h]\\
-[\vartheta(k,h),\alpha(l)]-[\vartheta(h,l),\alpha(k)]-[\vartheta(l,k),\alpha(h)].
\end{multline*}
$(\alpha,\beta)$ is a compatible pair then we can replace in $X$ the equalities  

$$\beta([\vartheta(k,h),l]) = [\beta(\vartheta(k,h)),l]+ [\vartheta(k,h)),\alpha(l)];$$
$$\beta([\vartheta(h,l),k]) = [\beta(\vartheta(h,l)),k]+ [\vartheta(h,l)),\alpha(k)];$$
$$\beta([\vartheta(l,k),h]) = [\beta(\vartheta(l,k)),h]+ [\vartheta(l,k)),\alpha(h)].$$

Hence
\begin{multline*}
X=[\beta(\vartheta(k,h)),l]+[\beta(\vartheta(h,l)),k]+[\beta(\vartheta(l,k)),h]\\
-[\vartheta(\alpha(k),h),l]-[\vartheta(\alpha(h),l),k]-[\vartheta(\alpha(l),k),h]\\
-[\vartheta(k,\alpha(h)),l]-[\vartheta(h,\alpha(l)),k]-[\vartheta(l,\alpha(k)),h].\\
\end{multline*} Again, by action definition we obtain
$$X=[(\alpha,\beta)\cdot\vartheta(h,l),k]+[(\alpha,\beta)\cdot\vartheta(l,k),h]+[(\alpha,\beta)\cdot\vartheta(k,h),l].$$
 So $(\alpha,\beta)\cdot\vartheta \in Z^2(K,I)$.
 
 Now suppose that $\vartheta \in B^2(K,I)$. Then there is a linear map $\nu:K \to I$ such that
 \begin{equation}
 {\label{1.4}} \vartheta(k,h)= \nu([k,h])-[\nu(k),h]-[k,\nu(h)].
 \end{equation}
 
Let $Y=(\alpha,\beta)\cdot \vartheta(k,h)$. By (\ref{1.4}) we have 
 
 $$Y=(\alpha,\beta)\cdot(\nu([k,h])-(\alpha,\beta)\cdot([\nu(k),h])-(\alpha,\beta)\cdot([k,\nu(h)]).$$ Using action difinition we have
\begin{multline*}
Y=\beta(\nu([k,h]))-\nu([\alpha(k),h])-\nu([k,\alpha(h)])\\
-\beta([\nu(k),h])+[\nu(\alpha(k)),h]+[\nu(k),\alpha(h)]\\
-\beta([k,\nu(h)])+[\alpha(k),\nu(h)]+[k,\nu(\alpha(h))],
\end{multline*} 
 we can use that $\alpha$ is a derivation and $(\alpha,\beta)$ is a compatible pair to conclude
\begin{multline*}
Y=\beta\nu([k,h])-\nu\alpha([k,h])\\
-[\beta\nu(k),h]-[\nu(k),\alpha(h)]+[\nu\alpha(k),h]+[\nu(k),\alpha(h)]\\
-[\beta(k),\nu(h)]-[k,\beta\nu(h)]+[\beta(k),\nu(h)]+[k,\nu\alpha(h)],
\end{multline*}  
 Hence, $$Y=(\beta\nu-\nu \alpha)[k,h]-[(\beta\nu-\nu \alpha)(k),h]+[k,(\beta\nu-\nu \alpha)(h)].$$
 If $T=\beta\nu-\nu \alpha:K \to I$ then $$(\alpha,\beta)\cdot \vartheta(k,h)=T([k,h])-[T(k),h]-[k,T(h)].$$ Therefore,  $(\alpha,\beta)\cdot\vartheta \in B^2(K,I)$.\begin{flushright}
 \cqd
 \end{flushright}
 
 This result allow us to define an action of $Comp(K,I)$ on $H^2(K,I)$: let $\vartheta \in Z^2(K,I)$ and $(\alpha, \beta) \in Comp(K,I)$. Define the action \begin{equation}{\label{acaoind}}
  (\alpha,\beta)\cdot (\vartheta +B^2(K,I))=((\alpha,\beta)\cdot \vartheta)+B^2(B,I).
 \end{equation}
 This is well defined by \textbf{Proposition \ref{inva}}.
 
  \begin{df}  Let $K$ be a Lie algebra and $I$ a $K$-module. Let $\vartheta \in Z^2(K,I)$ and consider the action of $Comp(K,I)$ on $H^2(K,I)$ defined in (\ref{acaoind}). Define the set of induced pairs of $Comp(K,I)$ by  $$Indu(K,I,\vartheta)=Ann_{Comp(K,I)}(\vartheta+B^2(K,I)).$$
 \end{df}

\vspace{0,5cm}

Let $K$ be a Lie algebra and $I$ a $K$-module. Let $\vartheta \in H^2(K,I)$ and suppose  that $I$, as ideal of $K_\vartheta$, it is invariant by derivation $d \in Der(K_\vartheta)$. Let $P_K:K_\vartheta\to K$ and $P_I:K_\vartheta\to I$ the natural projection of $K_\vartheta$ on $K$ and $K_\vartheta$ on $I$, respectively. Define the maps
\begin{itemize}{\label{w}}
\item $\alpha:K \to K$ by $\alpha(k)=P_K d(k)$, for all $k \in K$;
\item $\beta:I \to I$ by $\beta(a)=d(a),$ for all $a \in I$;
\item $\varphi:K \to I$ by $\varphi(k)=P_I d(k),$ for all $k \in K$.
\end{itemize} Then,   
\begin{equation}{\label{3}}
d(x+a)=\alpha(x)+\varphi(x)+\beta(a) \mbox{ for all } a \in I \mbox{ and } x \in K.
\end{equation}   
   Hence, $\beta \in Der(I)$, $\alpha \in Der(K)$ and $\varphi \in Hom(K,I).$

 We can see that $\beta$ is a derivation of $I$ bacause it is restriction of $d$ to $I$. 
 
 Let $x,y \in K$. By product definition on $K_\vartheta$ we have 
 $$d([x,y]_\vartheta)=d([x,y]_K+\vartheta(x,y)).$$ By decomposition showed in (\ref{3}) $$d([x,y]_\vartheta)=\alpha([x,y]_K)+\varphi([x,y]_K)+\beta(\vartheta(x,y)).$$ 
 
 We can calculate\begin{equation}\label{1.5}
 [d(x),y]_\vartheta+[x,d(y)]_\vartheta=[\alpha(x)+\varphi(x),y]+[x,\alpha(y)+\varphi(y)], 
 \end{equation} and use product definition of $K_\vartheta$ to get \begin{multline}{\label{1.6}}
 [d(x),y]_\vartheta+[x,d(y)]_\vartheta=[\alpha(x),y]_K+[x,\alpha(y)]_K+ \vartheta(\alpha(x),y) \\ +\vartheta(y,\alpha(x))+ [\varphi(x),\alpha(y)]-[\varphi(y),\alpha(x)].
\end{multline}
 
Comparing the components of $K$ in (\ref{1.5}) and (\ref{1.6}) we have $$\alpha([x,y]_K)=[\alpha(x),y]_K+[x,\alpha(y)]_K.$$ So $\alpha \in Der(K).$
 

\begin{prop}{\label{defphi2}} Let $K$ be a Lie algebra and $I$ a $K$-module. Let $\vartheta \in H^2(K,I)$ and suppose  that $I$, as ideal of $K_\vartheta$, it is invariant by derivations. From the decomposition showed in (\ref{3}) define $\phi:Der(K_\vartheta) \to Der(K)\oplus Der(I)$ by $\phi(d)=(\alpha,\beta)$. Then $\phi$ is a Lie algebra morphism.
\end{prop}

\prova Let $d,d' \in Der(K_\vartheta)$ and $x \in K$ such that  
$$ \begin{array}{rcl}
d(x+a) & = & \alpha(x)+ \varphi(x)+\beta(a) \\
d'(x+a) & = & \alpha'(x)+ \varphi'(x)+\beta'(x) ,
\end{array}$$ for all $x \in K$ and $a \in I$. Then,
$$\begin{array}{rcl}
 dd'(x)   & = & d(\alpha'(x)+\varphi'(x)) \\
           & = & \alpha \alpha'(x)+\varphi(\alpha'(x))+\beta'(\varphi'(x)). \\
\end{array}$$ Hence, $P_K dd'(x)=\alpha \alpha'(x).$ Analogously, $P_K d'd(x)=\alpha' \alpha'(x).$ So $P_K([d,d'])=[\alpha,\alpha']$. As $\beta$ and $\beta'$ are defined by $d$ and $d'$ restriction to $I$ then $P_I([d,d'])=[\beta,\beta']$. Therefore,
$$ \phi([d,d'])=([\alpha,\alpha'],[\beta, \beta'])=[(\alpha,\beta),(\alpha',\beta')]=[\phi(d),\phi(d')].$$ 
 \begin{flushright}
\cqd
\end{flushright}

\begin{teo}{\label{comp<im}} Let $K$ be a Lie algebra and $I$ a $K$-module. Let $\vartheta \in H^2(K,I)$ and suppose  that $I$, as ideal of $K_\vartheta$, it is invariant by derivations. Let $\phi: Der(K_\vartheta) \to Der(K)\oplus Der(I)$ given by $\phi(d)=(\alpha,\beta)$, defined in \textbf{Proposition {\ref{defphi2}}}. Then $Im(\phi)\leq Comp(K,I).$
\end{teo}
\prova  Let $(\alpha,\beta) \in Im(\phi)$. Then there is $d \in Der(K_\vartheta)$ such that $\phi(d)=(\alpha,\beta)$. If $k \in K$ and $a \in I$ then 
$$\begin{array}{rclc}
\beta([a,k]_\vartheta)  &= &  d([a,k]_\vartheta) & [a,k] \in I\\
&&& \\
&= &  [d(a),k]_\vartheta+[a,d(k)]_\vartheta & d \in Der(K_\vartheta)\\
&&& \\
&= &  [\beta(a),k]_\vartheta+[a,\alpha(k)+\varphi(k)]_\vartheta & \\
&&& \\
&= &  [\beta(a),k]_\vartheta+[a,\alpha(k)]_\vartheta & \mbox{ because $I$ is abelian }\\
\end{array}$$


\begin{flushright}
\cqd
\end{flushright}

 \begin{teo} Let $K$ be a Lie algebra and $I$ a $K$-module. Let $\vartheta \in H^2(K,I)$ and suppose  that $I$, as ideal of $K_\vartheta$, it is invariant by derivations. Let $\phi:Der(K_\vartheta) \to Der(K) \oplus Der(I)$ given by $\phi(d)=(\alpha,\beta)$. Then:
 
 \begin{enumerate}
 \item $Im(\phi)=Indu(K,I,\vartheta)$
  \item $ker(\phi) \cong Z^1(K,I)$
  \end{enumerate}
\end{teo}
\prova 1) Let $(\alpha,\beta) \in Indu(K,I,\vartheta)$. By definition  $$(\alpha,\beta)\cdot \vartheta=0  \mbox{ mod }  B^2(K,I).$$ Then, there is a linear map $T:K \to I$ such that for all $k,h \in K$ we have
 \begin{equation}{\label{2}}
  \vartheta(\alpha(k),h)+\vartheta(k,\alpha(h))+[T(k),h]-[T(h),k]=\beta(\vartheta(k,h))+T([k,h]).
\end{equation}

 Let $k \in K$, $a \in I$ and define the linear map $(\alpha,\beta)^*:K_\vartheta  \to  K_\vartheta$ by $$(\alpha,\beta)^*(k+a)=\alpha(k)+\beta(a)+T(k).$$
  Let's check that $(\alpha,\beta)^*$ is a derivation of $K_\vartheta$. Let $k+a,h+b \in K_\vartheta$. If  
  $$X= (\alpha,\beta)^*([k+a,h+b]_\vartheta)$$
   then
 $$\begin{array}{rcl}
X & = & (\alpha,\beta)^*([k,h]_K+\vartheta(k,h)+[a,h]-[b,k])\\
& = & \alpha([k,h]_K)+\beta(\vartheta(k,h))+\beta([a,h])-\beta([b,k])+ T([k,h]_K).\\
\end{array}$$
Now, let $$Y=[(\alpha+\beta)^*(k+a),h+b]_\vartheta+[k+a,(\alpha+\beta)^*(h+b)]_\vartheta. $$

We have
$$\begin{array}{rcl}
 [(\alpha+\beta)^*(k+a),h+b]_\vartheta& = & [\alpha(k)+\beta(a)+T(k),h+b]_\vartheta\\
 & = & [\alpha(k),h]_K+\vartheta(\alpha(k),h)+[\beta(a)+T(k),h]-[b,\alpha(k)] \\
 \end{array}$$ and
$$\begin{array}{rcl}
[k+a,(\alpha+\beta)^*(h+b)]_\vartheta & = & [k+a,\alpha(h)+\beta(b)+T(h)]\\
 & = & [k,\alpha(h)]_K+\vartheta(k,\alpha(h))+[a,\alpha(h)]-[\beta(b)+T(h),k]\\
 \end{array}$$ then
 \begin{multline*}
 Y=\alpha([k,h]_K)+ \vartheta(\alpha(k),h)+\vartheta(k,\alpha(h)) \\
+ [T(k),h]-[T(h),k] +[\beta(a),h])+[a,\alpha(h)] -[\beta(b),k]-[b,\alpha(k)].
 \end{multline*}

By compatible pair definition we get 
\begin{multline*}
Y=\alpha([k,h]_K)+ \vartheta(\alpha(k),h)+\vartheta(k,\alpha(h)) 
+\beta([a,h])-\beta([b,k])+[T(k),h]-[T(h),k].
\end{multline*}

By equation ({\ref{2}}) 
$$Y=\alpha([k,h]_K)+\beta(\vartheta(h,k))+T([k,h])+\beta([a,h])-\beta([b,k]).$$ 

As $X=Y$ then $(\alpha,\beta)^*$ is a derivation.

 Besides, observe that $P_K(\alpha,\beta)^*=\alpha$ and $P_I(\alpha,\beta)^*=\beta$. Hence $\phi((\alpha+\beta)^*)=\alpha+\beta$ , that is, $Indu(K,I,\vartheta) \subseteq Im(\phi)$. 
 
 Now, suppose that $(\alpha+\beta) \in Im(\phi)$. Then there is $d \in Der(K_\vartheta)$ such that $$\phi(d)=(\alpha+\beta).$$ By \textbf{Theorem \ref{comp<im}} we have $Im(\phi) \subseteq Comp(K,I)$. Then it is enough show that there is a linear map $T:K \to I$ such that the equation \ref{2} is satisfied.
 
  For each $k+a \in K_\vartheta$ we can use the decomposition defined in ({\ref{3}}) to write  $$d(k+a)=\alpha(k)+\varphi(k)+\beta(a).$$ By product definition in $K_\vartheta$ we get    
 $$\begin{array}{rcl}
 [d(k+a),h+b]_\vartheta & = &[\alpha(k)+\varphi(k)+\beta(a),h+b]_\vartheta\\
 & = &  [\alpha(k),h]_K+\vartheta(\alpha(k),h)+[\varphi(k)+\beta(a),h]-[b,\alpha(k)]\\
 \end{array}$$
 
 $$\begin{array}{rcl}
 [k+a,d(h+b)]_\vartheta& = &[k+a,\alpha(h)+\varphi(h)+\beta(b)]_\vartheta\\
 & = & [k,\alpha(h)]_K+\vartheta(k,\alpha(h))+[a,\alpha(h)]-[\varphi(h)+\beta(b),k] \\
 \end{array}$$
 
 $$\begin{array}{rcl}
d([k+a,h+b]_\vartheta) & = & d([k,h]_K+\vartheta(k,h)+[a,h]-[b,k])\\
 & = &  \alpha([k,h]_K)+\beta(\vartheta(k,h))+\beta([a,h])-\beta([b,k])+\varphi_d([k,h])\\
 \end{array}$$
 
 As $d$ is a derivation then we have equality  $$d[k+a,h+b]=[d(k)+a,h+b]=[k+a,d(h)+b].$$
  SO, 
  $$ \beta(\vartheta(k,h))+\varphi([k,h])
=\vartheta(\alpha(k),h)+[\varphi(k),h]+\vartheta(k,\alpha(h))-[\varphi(h),k]. 
 $$
  Therefore $T=\varphi$ satisfies the equation (\ref{2}) e $Im(\phi) \subseteq Indu(K,I,\vartheta).$
 \vspace{0,5cm}  

 2) Let $d \in ker(\phi)$. The decomposition showed in (\ref{3}) provide us $$d(k)=\varphi(k),k \in K.$$ Let $k,h \in K$. By derivation definition \begin{equation}{\label{4}}
 d([k,h]_\vartheta)=[d(k),h]_\vartheta+[k,d(h)]_\vartheta.
\end{equation}
 By product definition in $K_\vartheta$ we can write (\ref{4}) as
  $$d([k,h]_K+\vartheta(k,h))=[\varphi(k),h]_\vartheta+[k,\varphi(h)]_\vartheta.$$
 Because $d \in Ker(\phi)$ then (\ref{4}) it is equal to
  $$\varphi([k,h]_K)=[\varphi(k),h]_K-[\varphi(h),k]_K.$$ Hence, $\varphi \in Z^1(K,I)$. Now define $\sigma:ker(\phi) \to (Z^1(K,I),+)$ by $\sigma(d)=\varphi_d$ such that $\varphi_d(k)=d(k)$. Then $\sigma(Ker(\phi)) \subseteq Z^1(K,I)$. 
  
  Let $d,d' \in ker(\phi)$. Then $$\sigma(d+d')(k)=\varphi_{d+d'}(k)=(d+d')(k)=d(k)+d'(k)=\varphi(k)+\varphi'(k)=(\sigma(d)+\sigma(d'))(k).$$  So $\sigma$ it is group homomorfism.
 
 If $d,d' \in Ker(\phi)$ such that $\sigma(d)=\sigma(d')$ then $\varphi_d(k)=\varphi_{d'}(k),$ for all $k \in K$ and $d=d'$. Let $T \in Z^1(K,I)$ and define $d:K_\vartheta \to K_\vartheta$ by $$d(x+a)=T(x), x \in K, a \in I.$$ $d$ is a derivation because $$d([k+a,h+b]_\vartheta)=T([k,h]_K)$$ and $$[d(k+a),h+b]_\vartheta+[k+a,d(h+b)]_\vartheta=[T(k),h]_K+[k+T(h)]_K.$$
 It follows that $\sigma(d)=T$. Therefore, $\sigma$ is isomorphism\begin{flushright}
 \cqd
 \end{flushright}

\newpage
\

\bibliographystyle{plain}
\bibliography{mybib}











\end{document}
